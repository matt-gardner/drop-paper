Solving our dataset requires the joint efforts from text understanding and discrete reasoning. Existing reading comprehension studies mainly focused on the former one, especially with great progress in locating relevant information in the passage. On top of that, to answer more complex questions, a system should have the ability to do further computation or reasoning based on the superficial information it finds. As a preliminary attempt toward this goal, we propose an augmented QANet model, which equips the state-of-the-art reading comprehension system with additional abilities, including: (1) selecting spans from the question; (2) doing addition or subtraction over numbers; (3) predicting count numbers. The whole system can be trained end-to-end with a marginal likelihood loss.

\subsection{Basic QANet}

\subsection{}