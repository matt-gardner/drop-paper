%
% File naacl2019.tex
%
%% Based on the style files for ACL 2018 and NAACL 2018, which were
%% Based on the style files for ACL-2015, with some improvements
%%  taken from the NAACL-2016 style
%% Based on the style files for ACL-2014, which were, in turn,
%% based on ACL-2013, ACL-2012, ACL-2011, ACL-2010, ACL-IJCNLP-2009,
%% EACL-2009, IJCNLP-2008...
%% Based on the style files for EACL 2006 by 
%%e.agirre@ehu.es or Sergi.Balari@uab.es
%% and that of ACL 08 by Joakim Nivre and Noah Smith

\documentclass[11pt,a4paper]{article}
\usepackage[hyperref]{naaclhlt2019}
\usepackage{times}
\usepackage{latexsym}
\usepackage{microtype}
\usepackage{booktabs}
\usepackage{url}
\usepackage{eurosym}
\usepackage{todonotes}

%\aclfinalcopy % Uncomment this line for the final submission
%\def\aclpaperid{***} %  Enter the acl Paper ID here

%\setlength\titlebox{5cm}
% You can expand the titlebox if you need extra space
% to show all the authors. Please do not make the titlebox
% smaller than 5cm (the original size); we will check this
% in the camera-ready version and ask you to change it back.

\newcommand{\gabi}[1]{\todo[color=green!40]{\textbf{Gabi:} #1}}
\newcommand{\sameer}[1]{\todo[color=purple!40]{\textbf{sameer:} #1}}
\newcommand{\dheeru}[1]{\todo[color=olive!40]{\textbf{dheeru:} #1}}
\newcommand{\nascomment}[1]{\todo[color=blue!40]{\textbf{Noah:} #1}}

\newcommand{\figref}[1]{Figure~\ref{fig:#1}}
\newcommand{\secref}[1]{Section~\ref{sec:#1}}
\newcommand{\tabref}[1]{Table~\ref{tab:#1}}
\newcommand{\drop}[0]{DROP}


\title{DROP: A Reading Comprehension Benchmark \\ Requiring
Discrete Reasoning Over Paragraphs}

\author{First Author \\
  Affiliation / Address line 1 \\
  Affiliation / Address line 2 \\
  Affiliation / Address line 3 \\
  {\tt email@domain} \\\And
  Second Author \\
  Affiliation / Address line 1 \\
  Affiliation / Address line 2 \\
  Affiliation / Address line 3 \\
  {\tt email@domain} \\}

\date{}

\begin{document}
\maketitle
\begin{abstract}
  Reading comprehension has recently seen rapid progress, with systems matching humans on the most popular datasets for the task.  However, a large body of work has highlighted the brittleness of these systems, showing that there is much work left to be done.  We introduce a new reading comprehension benchmark, DROP, which requires {\bf D}iscrete {\bf R}easoning {\bf O}ver the content of {\bf P}aragraphs.  In this crowdsourced, adversarially-created, 50k-question benchmark, a system must resolve references in a question, perhaps to multiple input positions, and perform discrete operations over them (such as addition, counting, or sorting).  These operations require a much more comprehensive understanding of the content of paragraphs, as they remove the paraphrase-and-entity-typing shortcuts available in prior datasets.  We apply state-of-the-art methods from both the reading comprehension and semantic parsing literatures on this dataset and show that the best systems only achieve 28\% $F_1$ on our generalized accuracy metric, while expert human performance is 96.3\%.  We additionally present a new model that combines reading comprehension methods with simple numerical reasoning to achieve 41\% $F_1$.

\end{abstract}

\section{Introduction}
The task of \emph{reading comprehension}, where systems must understand a single passage of text well enough to answer arbitrary questions about it, has seen significant progress in the last few years.  This progress has been so effective that the most popular datasets available for this task have been solved (CITE BERT on SQuAD, Danqi Chen on CNN/Daily Mail).  Many follow-on tasks are currently being proposed, throwing in additional complexities around tracking conversational state (CITE CoQA, QuAC), requiring passage retrieval (CITE TriviaQA, HotpotQA), mismatched passages and questions (CITE DuoRC, NarrativeQA, SQuAD 2.0), integrating knowledge from external sources (CITE openbookqa), or a particular kind of ``multi-step'' reasoning over multiple documents (CITE WikiHop, MultiRC).  While these additional complexities are worth studying, they deviate from the original goal of reading comprehension, which remains far from solved.

\begin{figure}
\small
{\underline{Passage}: {\color{purple}In 2008, at another auction at Christies, Ulrich sold a 1982 Basquiat piece, Untitled (Boxer)}, for US \$13,522,500 to an anonymous telephone bidder. Another record price for a Basquiat painting was made {\color{olive}in 2007, when an untitled Basquiat work from 1981 sold at Sothebys in New York for US\$14.6 million}. {\color{olive}In 2012}, for the second year running, Basquiat was the most coveted contemporary (i.e. born after 1945) artist at auction, with {\color{olive}\euro 80 million in overall sales}. {\color{olive}That year}, his Untitled (1981), a painting of a haloed, black-headed man with a bright red skeletal body, depicted amid the artists signature scrawls, was  sold by \textbf{\textit{Robert Lehrman for {\color{olive}\$16.3 million}, well above its  \$12 million high estimate}}. A similar untitled piece, also undertaken in 1981 and formerly owned by the Israel Museum, {\color{purple}sold for \pounds312.92 million at Christies} London, setting a world auction record for Basquiats work. In 2013, Basquiats piece {\color{purple}Dustheads sold for \$48.8 million at Christies}. In 2016 an untitled piece sold at {\color{purple}Christies for \$57.3 million to a Japanese businessman} and collector, Yusaku Maezawa.
\\
{\color{olive} \underline{Question}: In which year was a Basquiat sold for a higher amount, 2007 or 2012?\\
    Answer: 2012\\
    BiDAF Answer. 1982
} \\
{\color{purple}
    \underline{Question}. How many Basquiats were sold at Christies?\\
    Answer. 4\\
    BiDAF Answer. \$48.8 million
}\\
\textbf{\textit{ \underline{Question}: How many more dollars was the Untitled (1981) painting sold for than the 12 million dollar estimation?\\
    Answer: 4300000\\
    BiDAF Answer. \$16.3 million
}}
}\\
    
%\hrule
%
%\tiny{
%\underline{Passage}: Coming off their season-sweep over the Browns, the Steelers flew to The Meadowlands for a Week 11 intraconference duel with the New York Jets.  In the first quarter, Pittsburgh surprisingly fell behind early as Jets QB Kellen Clemens completed a {\color{olive}1-yard TD pass to TE Chris Baker}, along with kicker Mike Nugent getting a 25-yard field goal.  In the second quarter, the Steelers got on the board with QB Ben Roethlisberger completing a {\color{olive}7-yard TD pass to WR Santonio Holmes}.  New York would end the half with Nugent kicking a 19-yard field goal. In the third quarter, {\color{teal}Pittsburgh tied the game with kicker Jeff Reed getting a 37-yard and a 33-yard field goal}. \textit{\textbf{ In the fourth quarter, the {\color{teal}Steelers took the lead with Reed getting a 48-yard field goal}.  However, the Jets managed to tie the game with Nugent kicking a 28-yard field goal}}.  In overtime, New York managed to pull off the upset as Nugent nailed the game-winning 38-yard field goal. Even worse, the Steelers highly ranked rushing defense allowed its first 100-yard rusher since 2005 (Edgerrin James of the Colts) as RB Thomas Jones got 30 carries for 117 yards. \\
%{\color{teal}
%    \underline{Question}: How many field goals did Pittsburgh Steelers make? \\
%    Answer: 3 \\
%    BiDAF Answer: 33-yard
%}\\
%{ \color{olive} 
%  \underline{Question}: Who scored the longest touchdown of the game? \\
%    Answer: Santonio Holmes \\
%    BiDAF Answer: Mike Nugent
%}\\
%{ \textit{\textbf{
%  \underline{Question}: In which quarter did both the teams make one field goal each? \\
%    Answer: fourth quarter \\
%    BiDAF Answer: third quarter, Pittsburgh tied the game with kicker Jeff Reed getting a 37-yard and a 33-yard field goal. In the fourth quarter
%}}}
%}
    \caption{Example passages, questions, and answers in DROP.}
    \label{fig:main_examples}
\end{figure}

We introduce a new, substantially more challenging reading comprehension dataset aimed at pushing the field towards more comprehensive analysis of paragraphs of text.  In this new benchmark, which we call DROP, a system is given a paragraph and a question and must perform some kind of {\bf D}iscrete {\bf R}easoning {\bf O}ver the text in the {\bf P}aragraph to obtain the correct answer.  These questions that require discrete reasoning (things like addition, sorting, or counting; see \figref{main_examples}) are inspired by the complex, compositional questions commonly found in the semantic parsing literature, and they force a system to understand the content of the paragraph well enough to reason over it, removing the paraphrase and entity typing shortcuts that were available in previous datasets.

We constructed this dataset through crowdsourcing, first collecting passages from Wikipedia that are easy to ask hard questions about, then encouraging crowd workers to produce challenging questions.  We additionally ran a baseline reading comprehension method (BiDAF; CITE) in the background as crowd workers were writing questions, requiring them to give questions that the baseline system could not correctly answer.  This resulted in a dataset of 54,000 questions from a variety of categories in Wikipedia, with a particular emphasis on sports game summaries and history passages.  The answers to the questions are required to be spans in the passage or question, numbers, or dates, which allows for easy and accurate evaluation metrics.

We analyze the resulting dataset to show what phenomena are present, and what challenges a system must overcome to correctly answer the questions.  We find that many questions combine complex question semantics with SQuAD-style argument finding; e.g., in the last question in \figref{main_examples}, BiDAF correctly finds the amount the painting sold for, but does not understand the question semantics and cannot perform the numerical reasoning required to answer the question.  Other questions, such as the second question in \figref{main_examples}, require finding all events in the passage that match a description in the question, then aggregating them somehow (in this instance, by counting them).  Very often some kind of entity coreference is required.

We used three types of systems to judge baseline performance on DROP: (1) heuristic, adversarial baselines, to check for biases in the data; (2) SQuAD-style reading comprehension methods; and (3) semantic parsers.  Of these, the reading comprehension methods perform the best, with our best baseline achieving 28\% $F_1$ on our generalized accuracy metric, while expert human performance is 96.7\%.

Finally, we contribute a new model for this task that combines simple numerical reasoning with standard reading comprehension methods, allowing the model to answer questions involving counting, addition and subtraction.  This model reaches 41\%, a 13\% absolute increase over the best baseline system.

The dataset, easily-extendable code for the baseline systems, and a leaderboard with a hidden test set can be found at https://withheld.for.review/.



\section{Related Work}
\gabi{Maybe better at the end, to get to the dataset faster}
TODO

Mention algebra word problems in here somewhere

\section{Reading Comprehension}


\section{DROP Data Collection}

We collect the dataset in three phases namely, extracting passage from Wikipedia, crowdsourcing question-answer pair collection and crowdsourcing a subset of questions in the dataset to obtain additional answers. 
\\
\textbf{Passage extraction}. To create questions that require arithmetic operations to answer, we sampled passages from Wikipedia  that have greater than twenty numbers in the passage. We oversampled from NFL game summaries and history articles in Wikipedia as they have a higher percentage of numerical values. \dheeru{Fix it later as it will be equally sampled}
\\
\textbf{Question-Answer collection}. We used Amazon Mechanical Turk to crowdsource the collection of question-answer pairs. We also setup a server hosting the BiDaF model to adversarialy create questions that can not be answered by a SQuAD-style model. The turkers were given 5 passages in each HIT and were asked to create a total of at least 12 questions. A question-answer pair was only accepted if the answer returned by the BiDaF model for that passage-question pair did not match the turker answer. Initially, we opened the HITs to all the turkers located in the United States. Gradually, as the turkers got acquainted with the task, we created a pool of good turkers and continued our dataset collection with these turkers. The turkers earned an average of \$10 per hour. Overall, the turkers seemed to enjoy the task and we received great reviews on TurkerView and in person.
\\
We provided sample questions from five chief database operation categories: Addition/Subtraction, Minimum/Maximum, Counting, Selection and Comparison to help the turkers focus on questions that need complex linguistic understanding and discrete reasoning. An answer could be of three types: span, date and number. A span is a continuous phrase from question or passage. An answer can have have multiple spans. The date-type answers were a part of history and other open-domain articles. A number-type answer is any real number. We restricted to ``how many \textit{[units]} " type questions for number answer. This helped us overlook specific units associated with each number making the evaluation simpler.
\\
We collected a total of 53,570 \dheeru{put the new value} question-answer pairs across all three domains: NFL, History and Open domain. The dataset was partitioned randomly into training (80\%), development(20\%) and test (20\%) sets, with mutually exclusive passages across the splits.


\textbf{Answer Validation}. 
To make the evaluation more robust, we additionally collected at least two answers for all the questions in development and test set via crowdsourcing. The turkers were shown passage-question pair and asked to answer the question. Additionally, they were given a checkbox to tick, if the question could not be answered. We found that the inter-annotator agreement was $\kappa=0.x$. \dheeru{Put correct value here and talk about it}. We, then, created a gold annotation set, with answers that at least two out of three turkers agreed on. \dheeru{What to do if there are full disagreement: check empirically and then write}. We also, collected expert annotations on 420 questions. The human performance measured against the gold set was 95.0\% accuracy (exact-match) and 96.3\% $F_1$\dheeru{update the numbers with new validation results}. x\% of questions were marked unanswerable and were removed from the dataset.






\section{DROP Data Analysis}
\begin{itemize}
    \item type of reasoning needed with examples to showcase the difficulty of task \ref{tab:interesting_example}
    \item question type distribution ( superlatives, count, and arithmetic etc.) - heuristic based so lower-bounds
    \item distribution of answer types - span, number, digit
\end{itemize}


\begin{table*}[t]
\centering
\small
\begin{tabular}{|p{1.5cm}|p{6.5cm}|p{3.5cm}|p{1.5cm}|p{1cm}|}
\hline
Reasoning & Passage & Question & Answer & BiDAF Answer\\
 \hline
 Addition or Subtraction & Before the UNPROFOR fully deployed, the HV clashed with an armed force of the RSK in the village of Nos Kalik, located in a pink zone near \u Sibenik, and captured the village at 4:45p.m. on {\color{teal}2 March 1992}. The JNA formed a battlegroup to counterattack the {\color{teal}next day}. & What date did the JNA form a battlegroup to counterattack after the village of Nos Kalik was captured?  & 3 March 1992 & 2 March 1992 \\ 
 \hline
 Comparison & In {\color{orange}1517, the seventeen-year-old King sailed to Castile}, where he was formally recognised as King of Castile. There, his Flemish court provoked much scandal, as de Cro\"y shamelessly sold government privileges for personal money and installed other Flemish nobles into government offices. {\color{orange}In May 1518, Charles traveled to Barcelona in Aragon}, where he would remain for nearly two years. & Where did Charles travel to first, Castile or Barcelona? & Castile & Aragon\\
 \hline
 Selection & In 1970, to commemorate the 100th anniversary of the founding of Baldwin City, {\color{purple}Baker University professor and playwright Don Mueller and Phyllis E. Braun, Business Manager, produced a musical play entitled The Ballad Of Black Jack} to tell the story of the events that led up to the battle. & Who was the University professor that helped produce The Ballad Of Black Jack, Ivan Boyd or Don Mueller? & Don Mueller & Baker\\
 \hline
 Complex Composition & James Douglas was the second {\color{olive}son of Sir George Douglas of Pittendreich, Master of Angus, and Elizabeth Douglas, daughter of David Douglas} of Pittendreich & Who was the grandfather of James Douglas? & David Douglas & Sir George Douglas\\
 \hline
\end{tabular}
\caption{Different reasoning required to answer questions in the dataset}
\label{tab:interesting_example}
\end{table*}

\section{Baseline Systems}

In this section, we test the performance of several strong baselines on the
\drop~ dataset.
First, in \secref{heuristics},
we try to identify and exploit annotation artifacts (following \cite{artifacts,max,bestpaperinemnlp})
that are bound to exist in \drop, even after our adversarial crowdsourcing protocol.
We then try two prominent approaches to QA on our dataset, evaluating
the state-of-the-art in semantic parsing (\secref{semparse}) and reading comprehension (\secref{rc}) to the \drop~ dataset.

\subsection{Heuristics and Adversarial Baselines}
\label{sec:heuristics}
%% TODO- GABI

%% \begin{itemize}
%% \item
%%   Look at the dev part, and come up with QA templates
%%   *which can be automatically detected*.
%% \item
%%   Assess their frequency on the dev set. How much of it do the chosen templates cover.
%%   (maybe this can be a part of the analysis section?)
%% \item
%%   Come up with heuristics for each QA template, probably based on frequencies in
%%   the dev set.
%%   How well do they do?
%% \end{itemize}

A recent line of work \cite{Gururangan:2018,Kaushik2018HowMR} has identified that popular synthetic NLP datasets
(such as SQuAD \cite{Rajpurkar2016SQuAD10} or SNLI \cite{Bowman2015ALA}) are prone to have artifacts and annotation biases which may be exploited by (advertently or inadvertently) by machine learning algorithms that
learn to pick up these artifacts as signal,
rather than actually learning something semantically meaningful.
These works have highlighted ways in which these artifcats can be exploited by models which explicitly focus on these biases, for example, by
 looking only at the question or the hypothesis in QA and entailment dataset, respectively.
Consequently, the resulting models were shown to be brittle and extremely sensitive
to minor data perturbations \cite{Glockner2018BreakingNS,others}.

While \drop~ was annotated in an adversarial setting
(see \secref{data_collection}) in an attempt to mitigate these effects, some
artifacts may still exist in the annotated data.
We try to estimate their extent by taking an adversarial
stance which tries to ``break'' \drop in some of the ways 
which were found to be successful on previous datasets.

\paragraph{Question-only baseline}

\paragraph{Paragraph-only baseline}

\paragraph{Majority baseline}


\subsection{Table Semantic Parser}
\label{sec:semparse}
%% \begin{itemize}
%%     \item Motivation - Predicate argument structure were developed in order to capture "meaning" of a sentence / proxy for information extraction, + brief discussion on the representation we experimented with.
%%     \item Predicate-argument structure representation.
%%     \item example.
%%     \item query language (mention that it's similar to other query languages).
%%     \item context representation.
%%     \item Model used 
%%     \item hyperparameters (tree depth,...)
%% \end{itemize}

Semantic parsing deals with translating a natural language utterance to
an executable formal language, which is expressible enough to allow certain arithmetic
operation (e.g., sorting or addition),  and
was shown to be particularly useful for constrained question-answering
over narrow domains \cite{Zelle1996LearningTP,Zettlemoyer2005LearningTM,Berant2014ModelingBP}.
Since many of \drop's questions require similar discrete reasoning, it is appealing
to port some of the successful work in semantic parsing to the \drop dataset.

\gabi{TODO: Pradeep, please take a look}
To that end, we will use the WikiTables parser \cite{Krishnamurthy2017neuralsp},
which populates a table from the input paragraph, and
parses the question as a query into that table.
The answer for the question is then deterministically computed given the predicted table
and query.

As \drop~ was constructed over paragraphs from various domains,
we experimented with several sentence representation formalisms which were
developed with the goal of capturing meaning in open-domain text.
Specifically, we will experiment with three prominent formalisms, each
representing different level of semantic granularity:
(1) syntactic dependencies \cite{sd}, which capture word-level relations,
(2) Open Information Extraction (Open IE; \cite{oie}), a shallow
form of semantic representation, which directly links predicates
with their arguments, abstracting away certain syntactic variations, and
(3) Semantic Role Labeling (SRL; \cite{srl}), which can be seen as an
augmentation of Open IE with predicate sense disambiguation and finer, predicate-specific
argument roles.

Each of these representations can be used to populate a table for WikiTables,
where lines correspond to predicate-argument structures,
and columns correspond to different argument types.
See \figref{semantic-parsing} for an example of the different predicate-argument
representations over an input sentence, and their corresponding tables.

Finally, we use the method described in \cite{Krishnamurthy2017neuralsp}
to train a solver to predict a logical form, and consequently an answer, from questions and tables.

\begin{table}[]
\begin{tabular}{@{}lllll@{}}
\toprule
Representation         & \#Rels & P & R & F1 \\ \midrule
Syntactic dependencies &             &           &        &    \\
Open IE                &             &           &        &    \\
Semantic Role Labeling  &             &           &        &    \\ \bottomrule
\end{tabular}
\end{table}


\subsection{SQuAD-style Reading Comprehension}
\label{sec:rc}
The success of reading comprehension on many existing benchmarks proved its effectiveness in aligning question with passage and extracting concise answer spans. 
% When constructing our dataset, we also allow a large portion of the answers (???\%) to be spans in the passage. However, we expect that such spans should not be extracted by simple lexical matching.
% Although our dataset is designed to avoid simple lexical matching, a large portion of the answers (???\%) are allowed to be spans in the passage, which could possibly be answered by SQuAD-style reading comprehension models. 
To test the real difficulty introduced by DROP that are different from existing benchmarks, we evaluate several representative SQuAD-style reading comprehension models on our dataset, including:
(1) \textbf{BiDAF} \cite{Seo2016BidirectionalAF}, which is the adversarial baseline we used in data construction; 
(2) \textbf{QANet} \cite{yu2018qanet}, which is currently the best-performing published model on SQuAD 1.1 without data augmentation or pre-training; 
(3) \textbf{QANet + ELMo}, which enhances the QANet model by concatenating the pre-trained ELMo \cite{peters2018elmo} representations to the original embeddings; 
(4) \textbf{BERT} \cite{Devlin2018BERTPO}, which recently achieved significant improvement on many NLP tasks by a novel way of pre-training, and we adopt their reading comprehension model for SQuAD. 
These four systems achieved different levels of performance on SQuAD 1.1, with the EM scores on development set as 66.8, 72.7, 78.1, 84.7 respecitvely.


The implementation details and the performance on squad should be mentioned here.

\section{Augmented QANet}
\label{sec:model}
Solving our dataset requires the joint efforts from text understanding and discrete reasoning. Existing reading comprehension studies mainly focused on the former one, especially with great progress in locating relevant information in the passage. On top of that, to answer more complex questions, a system should have the ability to do further computation or reasoning based on the superficial information it finds. As a preliminary attempt toward this goal, we propose an augmented QANet model, which equips the state-of-the-art reading comprehension system with additional abilities, including: (1) selecting spans from the question; (2) doing addition or subtraction over numbers; (3) predicting count numbers. The whole system can be trained end-to-end with a marginal likelihood loss.

\subsection{Basic QANet}

\subsection{}

\section{Results and Discussion}

TODO

\section{Conclusion}

TODO
\gabi{Consider as appendix: sample of the dataset (maybe using the dashboard?), snapshot of the annotation interface}

IGNORE, just for getting rid of build error\cite{Aho:72}


%\section*{Acknowledgments}
%
%The acknowledgments should go immediately before the references.  Do
%not number the acknowledgments section. Do not include this section
%when submitting your paper for review. \\

\bibliography{paper}
\bibliographystyle{acl_natbib}

\end{document}
