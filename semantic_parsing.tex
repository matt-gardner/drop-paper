%% \begin{itemize}
%%     \item Motivation - Predicate argument structure were developed in order to capture "meaning" of a sentence / proxy for information extraction, + brief discussion on the representation we experimented with.
%%     \item Predicate-argument structure representation.
%%     \item example.
%%     \item query language (mention that it's similar to other query languages).
%%     \item context representation.
%%     \item Model used 
%%     \item hyperparameters (tree depth,...)
%% \end{itemize}

Semantic parsing deals with translating natural language utterances to
into executable formal language. For tasks requiring answering questions in context, semantic parsing
has been used to translate natural language utterances into representations that can be
executed against some structured representation of the context, such as knowledge
graphs~\citep[among others]{Zettlemoyer2005LearningTM,berant2013semantic,Yin2017ASN,chen2011learning}.
This setup allows for complex reasoning over contextual knowledge, and it has
been successfully used in several natural language understanding
problems~\citep[among others]{}.
This paradigm was shown to particularly lend itself to constrained question-answering
over domain-specific datasets, such as geography (e.g., GeoQuery \cite{geoquery})
@@, or more recently @@ \gabi{more examples}.

In our setting, we will use sentence representation formalisms which were
developed with the goal of capturing meaning in open-domain text, to match \drop's
open-ended questions over from diverse domains,
for which traditional semantic parsing formalisms may not be expressive enough.
Specifically, we will experiment with three prominent formalisms, each
representing different semantic granularity level:
(1) syntactic dependencies \cite{sd}, which captures word-level relations, 
(2) Open Information Extraction (Open IE; \cite{oie}), a shallow
form of semantic representation, which links predicates
with their arguments, abstracting away certain syntactic variations, and
(3) Semantic Role Labeling (SRL; \cite{srl}), which can be seen as an
augmentation of Open IE with predicate sense disambiguation and finer, predicate-specific
argument roles.

Formally, we treat each of these three representations as a list of tuples, each
consisting of a single predicate and an arbitrary number of arguments. Both
predicate and arguments are spans from the raw sentence.

Given these list of tuples, we convert each representation into tabular format.
TODO-PRADEEP: Elaborate on conversion to table, and context augmentation.

See \figref{semantic-parsing} for an example of the different predicate-argument
representations over an input sentence.


\begin{table}[]
\begin{tabular}{@{}lllll@{}}
\toprule
Representation         & \#Rels & P & R & F1 \\ \midrule
Syntactic dependencies &             &           &        &    \\
Open IE                &             &           &        &    \\
Semantic Role Labeling  &             &           &        &    \\ \bottomrule
\end{tabular}
\end{table}
