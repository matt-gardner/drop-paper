%% \begin{itemize}
%%     \item Motivation - Predicate argument structure were developed in order to capture "meaning" of a sentence / proxy for information extraction, + brief discussion on the representation we experimented with.
%%     \item Predicate-argument structure representation.
%%     \item example.
%%     \item query language (mention that it's similar to other query languages).
%%     \item context representation.
%%     \item Model used 
%%     \item hyperparameters (tree depth,...)
%% \end{itemize}

Semantic parsing deals with translating natural language utterances to
into executable formal language. For tasks requiring answering questions in context, semantic parsing
has been used to translate natural language utterances into representations that can be
executed against some structured representation of the context, such as knowledge
graphs~\citep[among others]{Zettlemoyer2005LearningTM,berant2013semantic,Yin2017ASN,chen2011learning}.
This setup allows for chaining of discrete operators for reasoning over contextual knowledge.
%and it has
%been successfully used in several natural language understanding
%problems~\citep[among others]{}.
%This paradigm was shown to particularly lend itself to constrained question-answering
%over domain-specific datasets, such as geography (e.g., GeoQuery \cite{geoquery})
%@@, or more recently @@ \gabi{more examples}.

Since many of \drop's questions require similar discrete reasoning, it is appealing
to port some of the successful work in semantic parsing to the \drop~ dataset.
To that end, we will use the grammar-constrained semantic parsing model
built by~\cite{Krishnamurthy2017neuralsp} (KDG Parser henceforth). 
Since semantic parsers assume the context is structured, a prerequisite for using the
KDG parser, or any semantic parser for that matter,
for \drop, is to represent paragraphs as structured contexts. We do so by running various structured prediction models
on the paragraphs.

\paragraph{Extracting predicate argument structures} As \drop~ was constructed over paragraphs from various domains,
we experimented with several sentence representation formalisms which were
developed with the goal of capturing meaning in open-domain text.
Specifically, we will experiment with three prominent formalisms, each
representing a different level of semantic granularity:
(1) syntactic dependencies \cite{Marneffe2008TheST}, which capture word-level relations,
(2) Open Information Extraction \citep[Open IE]{Banko2007OpenIE}, a shallow
form of semantic representation, which directly links predicates
with their arguments, abstracting away certain syntactic variations, and
(3) Semantic Role Labeling \citep[SRL]{SRL}, which can be seen as an
augmentation of Open IE with predicate sense disambiguation and finer, predicate-specific
argument roles. Since the KDG parser was built for
the \textsc{WikiTableQuestions} dataset~\cite{Pasupat2015CompositionalSP},
where the context is in a tabular format, 
we transform the output of the structured prediction models into a table.
We represent each predicate-argument structure as a row in a table, where the columns correspond
to the argument types.

\paragraph{Augmenting predicate argument structures} To facilitate the production of numbers, dates, and
other entities, we augment the predicate argument structures with the following additional information.
We run a named entity tagger (TODO: cite Spacy's tagger?) on all sentences, and extract  
See \figref{semantic-parsing} for an example of the different predicate-argument
representations over an input sentence, and their corresponding tables.

\paragraph{Logical form language}

\paragraph{Training the parser}
Finally, we use the method described in \cite{Krishnamurthy2017neuralsp}
to train a solver to predict a logical form, and consequently an answer, from questions and tables.

\begin{table}[]
\begin{tabular}{@{}lllll@{}}
\toprule
Representation         & \#Rels & P & R & F1 \\ \midrule
Syntactic dependencies &             &           &        &    \\
Open IE                &             &           &        &    \\
Semantic Role Labeling  &             &           &        &    \\ \bottomrule
\end{tabular}
\end{table}
